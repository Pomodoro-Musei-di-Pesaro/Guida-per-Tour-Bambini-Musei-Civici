\documentclass[hidelinks,12pt,a4paper]{exam}
\usepackage[italian]{babel}
\usepackage[utf8]{inputenc}
\usepackage{fourier} 

% To avoid GitHub Action error
\usepackage{hyperref}

% To clone parts of text
\usepackage{clipboard}

% For Creating a table
\usepackage{tabularx}

% Adjust paragraph.
\usepackage{changepage}

% Color text.
\usepackage{xcolor}

% Stop hyphenation
\usepackage[none]{hyphenat}

% License
\usepackage[
type={CC},
modifier={by-nc-sa},
version={4.0},
]{doclicense}

% Command to create fake sections
\newcommand{\fakesection}[1]{%
	\par\refstepcounter{section}% Increase section counter
	\sectionmark{#1}% Add section mark (header)
	\addcontentsline{toc}{section}{\protect\numberline{\thesection}#1}% Add section to ToC
}

% Create a special enumerate enviroment
\usepackage{enumitem}% http://ctan.org/pkg/enumitem
\newlist{enumAnswers}{enumerate}{1}
\setlist[enumAnswers,1]{label=\textbf{Risposta domanda \arabic*. }}

\begin{document}
	
	\title{\textbf{Test per Bambini}}
	\author{Francesco Rombaldoni}
	\date{}
	\maketitle
	
	\newpage
	\pagestyle{plain}
	\tableofcontents
	\newpage
	
	\section{Come somministrare e correggere il Test}
	\begin{center}
		\textbf{I consigli sono rivolti agli Operatori.}
	\end{center}
	
	Questo test serve per valutare la comprensione del tour guidato ai Musei Civici da parte dei bambini, per questo motivo non è necessario somministrare il compito ai singoli bambini, ma per favorire il lavoro di gruppo è consigliabile dividere i bambini in gruppi di massimo tre elementi (in questo caso il campo "nome" diventa il nome del gruppo, oppure i nomi dei componenti della squadra). Il tempo consigliato per completare il test è di circa quindici minuti.\\
	Scaduto il tempo, raccogliere i compiti ed iniziare la correzione, per questa fase si consiglia di segnare con una penna rossa gli errori commessi e con la stessa scrivere nel campo "voto" la valutazione del test. Per segnare il voto è preferibile usare una valutazione descrittiva (come la scala: sufficiente, discreto, buono, distinto, ottimo) piuttosto che una valutazione numerica (esempio quella decimale), in modo che i bambini non facciano grandi confronti tra loro, in modo da mantenere un clima più sereno.\\
	Le ultime quattro domande aperte non sono da correggere in quanto servono per capire le preferenze dei bambini.\\
	In caso di dubbi per la correzione, tenere una copia digitale di questo documento consultabile dallo "smartphone".
	
	\begin{center}
		\large{\textbf{Tabella per la correzione}}\\
		\bigskip
		
		\begin{tabularx}{0.5\textwidth} { 
				| >{\raggedright\arraybackslash}X 
				| >{\centering\arraybackslash}X | }
			\hline
			\textbf{Descrizione} & \textbf{Voto} \\
			\hline
			Sono stati commessi massimo due errori & Ottimo\\
			\hline
			Sono stati commessi dai tre ai cinque errori & Distinto\\
			\hline
			Sono stati commessi dai sei agli otto errori & Buono\\
			\hline
			Sono stati commessi dai nove ai dieci errori & Discreto\\
			\hline
			Sono stati commessi numerosi errori & Sufficiente\\
			\hline
		\end{tabularx}
	\end{center}
	
	\newpage
	
	\fakesection{Test}
	% Remove page numbers
	\pagestyle{empty}
	
	\fboxrule=2pt
	\centerline{
		\fbox{
			\begin{minipage}{\linewidth}
				\centering{\textbf{Nome:}\line(1,0){200}}\\
				\textbf{Data:}\line(1,0){50}
				\hfill
				\textbf{Voto:}\line(1,0){50}
			\end{minipage}
		}
	}
	
	\bigskip
	\begin{center}
		\textcolor{red}{Dopo aver letto la domanda, colorale il cerchio accanto la risposta giusta tra quelle proposte.
		\textbf{{\large Attenzione}, alcune domande possono avere più di una risposta corretta.}}
	\end{center}
	
	% Starting Multiple Choice Questions here
	\begin{questions}
		\Copy{questions}
		{
			
			\question Cosa accadde all'autore della Medusa durante il trasporto dell'opera in occasione della Binnale di Arti Decorative di Monza?
			\begin{checkboxes}
				\choice L'opera rotolò scheggiandosi ai bordi
				\CorrectChoice la Medusa gli crollò addosso schiacciandolo con il suo peso
				\choice uno dei serpenti appartenenti alla chioma si ruppe
			\end{checkboxes}
		
			\question Come venne trasportata la Pala Bellini fino a Pesaro?
			\begin{checkboxes}
				\choice In aereo
				\choice In treno
				\CorrectChoice Via mare smontata in varie parti
			\end{checkboxes}
		
			\question Cosa piaceva a Rossini nei dipinti?
			\begin{checkboxes}
				\CorrectChoice La Presenza del colore rosso,di oro e gioielli
				\choice Le rappresentazioni di belle donne, il cibo ed i boccali di vini e liquori
				\choice Le rappresentazioni di carattere religioso perché molto devoto alla preghiera
			\end{checkboxes}
		
			\question Che animale ha un insolito colore nero nella ceramica Leda e il Cigno?
			\begin{checkboxes}
				\choice Un Gufo
				\CorrectChoice Uno Scoiattolo
				\choice Il Cigno
			\end{checkboxes}
		
			\question Che tipo di decorazioni ci sono nello Specchio in Vetro di Murano?\\
			\begin{checkboxes}
				\CorrectChoice Incisioni d'uva in argento
				\choice decorazioni con animali
				\CorrectChoice Decorazioni di fiori in vetro
				\choice incisioni di paesaggi
			\end{checkboxes}
		
			\question Che animale ha ucciso Adone durante la caccia?
			\begin{checkboxes}
				\choice Un Lupo
				\CorrectChoice Un Cinghiale
				\choice Una Volpe
			\end{checkboxes}
			
			\question Di che malattia soffriva la Marchesa Toschi-Mosca?
			\begin{checkboxes}
				\CorrectChoice Insonnia
				\choice Sonnambulismo
				\choice Mal di stomaco
			\end{checkboxes}
		
			\question Che Paesaggio è raffigurato nella coppia di Stipi della Marchesa Toschi-Mosca?
			\begin{checkboxes}
				\choice Venezia
				\choice Firenze
				\CorrectChoice Roma
			\end{checkboxes}
		
			\question Che tipo di scene sono dipinte nel Mercato di Aureliano Milani?
			\begin{checkboxes}
				\CorrectChoice Di vita quotidiana
				\choice Di vita di Corte, nobiliare
				\choice di natura ecclesiastica, religiosa
			\end{checkboxes}
		}
	\end{questions}

	% End Multiple Choice Questions
	
	% Starting Open-Ended Questions here
	
	\begin{enumerate}[start=10] %This is a simple solution to have valid question numbers

		 \item Quale tra Sant'Ambrogio in Trono e Trompe l'Oelil con Sonetto ti sembra più realistico come quadro? \\Motiva la tua risposta.\\
					\line(1,0){\linewidth}\\
					\line(1,0){\linewidth}\\
		
		\item Che opera pittorica ti è piaciuta di più all'interno del Museo?\\
					\line(1,0){\linewidth}\\
					\line(1,0){\linewidth}\\

		\item Quale delle stanze del Museo che hai visitato ti ricordi meglio e perchè?\\
					\line(1,0){\linewidth}\\
					\line(1,0){\linewidth}\\
					
	\item Quale opera ceramica ti ha colpito di più?\\
					\line(1,0){\linewidth}\\
					\line(1,0){\linewidth}\\
					
	\item Se hai visto gli spettacoli della Sonosfera quale dei due video ti è piaciuto di più?\\
					\line(1,0){\linewidth}\\
					\line(1,0){\linewidth}\\
		
		
	\end{enumerate}
	% End Open-Ended Questions
	
	\newpage
	\section{Risposte Test}
	
	\pagestyle{plain}
	
	% Multiple Choice Questions Answers (auto-generated)
	\begin{questions}
		\printanswers
		\Paste{questions}
	\end{questions}

	% Open-ended Questions Answers
	\begin{adjustwidth}{35mm}{}
		\begin{enumAnswers}[start=10]
			\item Trompe l'Oelil con Sonetto è più realistico come quadro, in quanto fa uso della prospettiva, cosa che manca nel dipinto di Sant'Ambrogio in Trono.
		\end{enumAnswers}
	\end{adjustwidth}
	
	\vspace*{\fill}
	% Print license shield
	\doclicenseThis
\end{document}
