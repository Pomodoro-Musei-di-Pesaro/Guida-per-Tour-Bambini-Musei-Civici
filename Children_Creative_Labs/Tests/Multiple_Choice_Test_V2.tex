\documentclass[hidelinks,12pt,a4paper]{exam}
\usepackage[italian]{babel}
\usepackage[utf8]{inputenc}
\usepackage{fourier} 


% License
\usepackage[
type={CC},
modifier={by-nc-sa},
version={4.0},
]{doclicense}

\begin{document}

	\title{\textbf{Test per i bambini}}
	\author{Alice Balestieri\\Francesco Rombaldoni}
	\date{}
	\maketitle
	
	\newpage
	
	\tableofcontents
	\newpage
	
	\section{Come somministrare e correggere il Test}
	\begin{center}
		\textbf{I consigli sono rivolti agli operatori.}
	\end{center}
	
	Questo test serve per valutare la comprensione del tour guidato ai Musei Civici da parte dei bambini, per questo motivo non è necessario somministrare il compito ai singoli bambini, ma per favorire il lavoro di gruppo è consigliabile di dividere i bambini in gruppi di massimo tre elementi (in questo caso il campo "nome" diventa il nome del gruppo, oppure i nomi dei componenti della squadra). Il tempo consigliato per completare il test è di circa un'ora.\\
	Scaduto il tempo raccogliere i compiti ed iniziare la correzione, per questa fase si consiglia in particolare di segnare con una penna rossa gli errori commessi e con la stessa scrivere nel campo "voto" la valutazione del test. Per segnare il voto si è preferibile usare una valutazione descrittiva piuttosto che una valutazione numerica (esempio quella decimale), in modo che i bambini non possano fare grandi confronti tra di loro, in modo da tenere un clima più sereno anche al fronte di un voto che potrebbe creare disparità.
	
	
	\newpage
	
	\fboxrule=2pt
	\centerline{
		\fbox{
		\begin{minipage}{\linewidth}
			\centering{\textbf{Nome:}\line(1,0){200}}\\
			\textbf{Data:}\line(1,0){50}
			\hfill
			\textbf{Voto:}\line(1,0){50}
		\end{minipage}
		}
	}
	
	% Starting questions here
	
	\begin{questions}
		
		\question Which of these famous physicists published a paper on Brownian Motion?
		\begin{checkboxes}
			\choice Stephen Hawking 
			\CorrectChoice Albert Einstein
			\choice Emmy Noether
			\choice I don't know
		\end{checkboxes}
		
	\end{questions}
	
	\printanswers

\end{document}