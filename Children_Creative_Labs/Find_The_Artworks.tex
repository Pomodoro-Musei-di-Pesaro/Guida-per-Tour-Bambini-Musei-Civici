\documentclass[hidelinks,12pt,a4paper]{article}
\usepackage[italian]{babel}
\usepackage[utf8]{inputenc}
\usepackage{fourier} 

% To avoid GitHub Action error
\usepackage{hyperref}

% Images
\usepackage{graphicx}
\usepackage{caption}
\usepackage{subcaption}
\usepackage{float}
\graphicspath{ {../Images} }

% Stop hyphenation
\usepackage[none]{hyphenat}

% Adjust paragraph.
\usepackage{changepage}

% Write over image.
\usepackage{tikz}

% Adjust image dim.
\usepackage{adjustbox}

% Minipages in the same line
\usepackage{tabularx}

% License
\usepackage[
type={CC},
modifier={by-nc-sa},
version={4.0},
]{doclicense}

%Comand to create cards
\newcommand{\cards}[2]{  % First Argument -> includegraphics entire string 
												% Second Argumenti -> Image caption

%---------- Begin page ----------
\begin{adjustwidth}{-30mm}{-30mm}
	
	% Top section
		\begin{tabularx}{\textwidth}{XX}
		{
			\begin{tikzpicture}
				\node[draw,dashed]
				{
					\fboxrule=0pt{
						\setlength{\fboxsep}{0pt}\fbox{
							
							% Minipage "container"
							\begin{minipage}[t][8.3cm][t]{8.9cm}
								\vspace*{-1mm}
								\fbox{
									\begin{minipage} [l] [\dimexpr \linewidth \relax] [t] {\dimexpr 0.95\linewidth \relax}
										\medskip
										\centering
										\begin{tikzpicture}
											\node[anchor=south west,inner sep=0] (image) at (0,0) {#1};
											\begin{scope}[x={(image.south east)},y={(image.north west)}]
												\draw[draw=none,fill=white] (0,0) rectangle (0.21,0.31);
												\draw[draw=none,fill=white] (0.2,0) rectangle (0.51,0.11);
												\draw[draw=none,fill=white] (0,0.3) rectangle (0.11,0.41);
												\draw[draw=none,fill=white] (0.4,0.1) rectangle (0.51,0.31);
												\draw[draw=none,fill=white] (0.5,0.1) rectangle (0.91,0.21);
												\draw[draw=none,fill=white] (0.7,0) rectangle (0.81,0.51);
												\draw[draw=none,fill=white] (0.6,0.2) rectangle (0.71,0.31);
												\draw[draw=none,fill=white] (0.3,0.2) rectangle (0.41,0.51);
												\draw[draw=none,fill=white] (0.5,0.3) rectangle (0.61,0.41);
												\draw[draw=none,fill=white] (0.9,0.3) rectangle (1.01,0.61);
												\draw[draw=none,fill=white] (0,0.6) rectangle (0.91,0.71);
												\draw[draw=none,fill=white] (0.1,0.5) rectangle (0.21,0.81);
												\draw[draw=none,fill=white] (0.2,0.5) rectangle (0.31,1);
												\draw[draw=none,fill=white] (0.4,0.4) rectangle (0.51,1);
												\draw[draw=none,fill=white] (0,0.9) rectangle (1.01,1);
												\draw[draw=none,fill=white] (0,0.8) rectangle (0.11,0.91);
												\draw[draw=none,fill=white] (0.3,0.7) rectangle (0.41,0.91);
												\draw[draw=none,fill=white] (0.5,0.5) rectangle (0.61,0.61);
												\draw[draw=none,fill=white] (0.6,0.4) rectangle (0.71,0.61);
												\draw[draw=none,fill=white] (0.8,0.5) rectangle (0.91,0.91);
												\draw[draw=none,fill=white] (0.6,0.8) rectangle (1.01,0.91);
											\end{scope}
										\end{tikzpicture}
										
									\end{minipage}
									
								}
								
								\vspace*{-2.8cm}
								
								\fboxrule=4pt{
									\framebox[\textwidth]{\rule{0pt}{55pt}}}
								
							\end{minipage}
							
						}
					}
				};
			\end{tikzpicture}
			
		}&{
		
			\hspace*{3.cm}
			\begin{tikzpicture}
				\node[draw,dashed]
				{
					\fboxrule=0pt{
						\setlength{\fboxsep}{0pt}\fbox{
						
							% Minipage "container"
							\begin{minipage}[t][8.3cm][t]{8.9cm}
								\vspace*{-1mm}
								\fbox{
									\begin{minipage} [l] [\dimexpr \linewidth \relax] [t] {\dimexpr 0.95\linewidth \relax}
										\medskip
										\centering
										\begin{tikzpicture}
											\node[anchor=south west,inner sep=0] (image) at (0,0) {#1};
											\begin{scope}[x={(image.south east)},y={(image.north west)}]
												\draw[draw=none,fill=white] (0.1,0.3) rectangle (0.71,0.51);
												\draw[draw=none,fill=white] (0,0.1) rectangle (0.81,0.21);
												\draw[draw=none,fill=white] (0.1,0.3) rectangle (0.71,0.51);
												\draw[draw=none,fill=white] (0,0) rectangle (0.11,0.31);
												\draw[draw=none,fill=white] (0.1,0.1) rectangle (0.21,0.31);
												\draw[draw=none,fill=white] (0.2,0) rectangle (0.31,0.21);
												\draw[draw=none,fill=white] (0.3,0) rectangle (0.41,0.61);
												\draw[draw=none,fill=white] (0.5,0) rectangle (0.61,0.21);
												\draw[draw=none,fill=white] (0.6,0.1) rectangle (0.71,1);
												\draw[draw=none,fill=white] (0.1,0.3) rectangle (0.71,0.51);
												\draw[draw=none,fill=white] (0.7,0) rectangle (0.81,0.11);
												\draw[draw=none,fill=white] (0.9,0.1) rectangle (1.01,0.21);
												\draw[draw=none,fill=white] (0.8,0.3) rectangle (0.91,1);
												\draw[draw=none,fill=white] (0,0.4) rectangle (1.01,0.51);
												\draw[draw=none,fill=white] (0,0.4) rectangle (0.11,0.71);
												\draw[draw=none,fill=white] (0.5,0.3) rectangle (0.61,0.61);
												\draw[draw=none,fill=white] (0.6,0.6) rectangle (1.01,0.71);
												\draw[draw=none,fill=white] (0.6,0.8) rectangle (1.01,1);
												\draw[draw=none,fill=white] (0,0.6) rectangle (0.31,0.71);
												\draw[draw=none,fill=white] (0.1,0.7) rectangle (0.41,0.81);
												\draw[draw=none,fill=white] (0.4,0.6) rectangle (0.51,0.71);
												\draw[draw=none,fill=white] (0.5,0.7) rectangle (0.71,0.81);
												\draw[draw=none,fill=white] (0,0.8) rectangle (0.11,1);
												\draw[draw=none,fill=white] (0,0.9) rectangle (0.41,1);
												\draw[draw=none,fill=white] (0.3,0.8) rectangle (0.51,0.91);
												\draw[draw=none,fill=white] (0.5,0.9) rectangle (0.81,1);
											\end{scope}
										\end{tikzpicture}
									
									\end{minipage}
									
								}
							
								\vspace*{-2.8cm}
							
								\fboxrule=4pt{
									\framebox[\textwidth]{\rule{0pt}{55pt}}}
							
							\end{minipage}
						
						}
					}
				};
			\end{tikzpicture}
		}
		\end{tabularx}
		
	% Bottom section
			\begin{tabularx}{\textwidth}{XX}
		{
			\begin{tikzpicture}
				\node[draw,dashed]
				{
					\fboxrule=0pt{
						\setlength{\fboxsep}{0pt}\fbox{
							
							% Minipage "container"
							\begin{minipage}[t][8.3cm][t]{8.9cm}
								\vspace*{-1mm}
								\fbox{
									\begin{minipage} [l] [\dimexpr \linewidth \relax] [t] {\dimexpr 0.95\linewidth \relax}
										\medskip
										\centering
										\begin{tikzpicture}
											\node[anchor=south west,inner sep=0] (image) at (0,0) {#1};
											\begin{scope}[x={(image.south east)},y={(image.north west)}]
												\draw[draw=none,fill=white] (0,0) rectangle (0.51,0.21);
												\draw[draw=none,fill=white] (0.4,0.1) rectangle (0.81,0.21);
												\draw[draw=none,fill=white] (0.2,0) rectangle (0.31,0.31);
												\draw[draw=none,fill=white] (0.3,0) rectangle (0.41,0.41);
												\draw[draw=none,fill=white] (0.5,0.1) rectangle (0.61,0.31);
												\draw[draw=none,fill=white] (0.5,0.1) rectangle (0.61,0.31);
												\draw[draw=none,fill=white] (0.6,0) rectangle (0.71,0.21);
												\draw[draw=none,fill=white] (0.8,0) rectangle (1.01,0.11);
												\draw[draw=none,fill=white] (0.9,0) rectangle (1.01,0.21);
												\draw[draw=none,fill=white] (0,0.3) rectangle (0.31,0.41);
												\draw[draw=none,fill=white] (0.1,0.4) rectangle (0.31,0.61);
												\draw[draw=none,fill=white] (0,0.5) rectangle (0.41,0.61);
												\draw[draw=none,fill=white] (0.4,0.3) rectangle (0.51,0.51);
												\draw[draw=none,fill=white] (0.4,0.4) rectangle (0.61,0.51);
												\draw[draw=none,fill=white] (0.6,0.3) rectangle (1.01,0.41);
												\draw[draw=none,fill=white] (0.5,0.5) rectangle (0.81,0.91);
												\draw[draw=none,fill=white] (0.7,0.3) rectangle (0.81,0.61);
												\draw[draw=none,fill=white] (0.7,0.5) rectangle (1.01,0.61);
												\draw[draw=none,fill=white] (0.9,0.5) rectangle (1.01,0.71);
												\draw[draw=none,fill=white] (0,0.7) rectangle (0.11,0.91);
												\draw[draw=none,fill=white] (0.2,0.7) rectangle (0.41,1);
												\draw[draw=none,fill=white] (0.1,0.9) rectangle (0.31,1);
												\draw[draw=none,fill=white] (0.3,0.6) rectangle (0.51,0.71);
												\draw[draw=none,fill=white] (0.4,0.6) rectangle (0.51,0.81);
												\draw[draw=none,fill=white] (0.5,0.9) rectangle (0.71,1);
												\draw[draw=none,fill=white] (0.8,0.7) rectangle (0.91,1);
												\draw[draw=none,fill=white] (0.8,0.8) rectangle (1.01,0.91);
											\end{scope}
										\end{tikzpicture}
										
									\end{minipage}
									
								}
								
								\vspace*{-2.8cm}
								
								\fboxrule=4pt{
									\framebox[\textwidth]{\rule{0pt}{55pt}}}
								
							\end{minipage}
							
						}
					}
				};
			\end{tikzpicture}
			
		}&{
			
			\hspace*{3.cm}
			\begin{tikzpicture}
				\node[draw,dashed]
				{
					\fboxrule=0pt{
						\setlength{\fboxsep}{0pt}\fbox{
							
							% Minipage "container"
							\begin{minipage}[t][8.3cm][t]{8.9cm}
								\vspace*{-1mm}
								\fbox{
									\begin{minipage} [l] [\dimexpr \linewidth \relax] [t] {\dimexpr 0.95\linewidth \relax}
										\medskip
										\centering
										\begin{tikzpicture}
											\node[anchor=south west,inner sep=0] (image) at (0,0) {#1};
											\begin{scope}[x={(image.south east)},y={(image.north west)}]
												\draw[draw=none,fill=white] (0,0) rectangle (0.11,0.41);
												\draw[draw=none,fill=white] (0,0.1) rectangle (0.21,0.21);
												\draw[draw=none,fill=white] (0.3,0) rectangle (0.41,0.41);
												\draw[draw=none,fill=white] (0.2,0) rectangle (0.41,0.11);
												\draw[draw=none,fill=white] (0.3,0.1) rectangle (0.51,0.21);
												\draw[draw=none,fill=white] (0.5,0) rectangle (0.71,0.11);
												\draw[draw=none,fill=white] (0.8,0) rectangle (1.01,0.21);
												\draw[draw=none,fill=white] (0.9,0) rectangle (1.01,0.41);
												\draw[draw=none,fill=white] (0.8,0.3) rectangle (0.91,1);
												\draw[draw=none,fill=white] (0.7,0.2) rectangle (0.81,0.31);
												\draw[draw=none,fill=white] (0.2,0.2) rectangle (0.31,0.81);
												\draw[draw=none,fill=white] (0.1,0.3) rectangle (0.31,0.51);
												\draw[draw=none,fill=white] (0.5,0.3) rectangle (0.61,0.91);
												\draw[draw=none,fill=white] (0.5,0.3) rectangle (0.71,0.51);
												\draw[draw=none,fill=white] (0.4,0.4) rectangle (0.81,0.51);
												\draw[draw=none,fill=white] (0,0.5) rectangle (0.11,1);
												\draw[draw=none,fill=white] (0,0.7) rectangle (0.21,0.91);
												\draw[draw=none,fill=white] (0.2,0.6) rectangle (0.61,0.81);
												\draw[draw=none,fill=white] (0.2,0.5) rectangle (0.41,0.61);
												\draw[draw=none,fill=white] (0.5,0.6) rectangle (0.71,0.71);
												\draw[draw=none,fill=white] (0.2,0.9) rectangle (0.51,1);
												\draw[draw=none,fill=white] (0.4,0.7) rectangle (0.51,1);
												\draw[draw=none,fill=white] (0.7,0.5) rectangle (1.01,0.61);
												\draw[draw=none,fill=white] (0.7,0.7) rectangle (1.01,0.91);
												\draw[draw=none,fill=white] (0.6,0.9) rectangle (0.91,1);
											\end{scope}
										\end{tikzpicture}
										
									\end{minipage}
									
								}
								
								\vspace*{-2.8cm}
								
								\fboxrule=4pt{
									\framebox[\textwidth]{\rule{0pt}{55pt}}}
								
							\end{minipage}
							
						}
					}
				};
			\end{tikzpicture}
		}
	\end{tabularx}
		
\end{adjustwidth}

\vspace*{\fill}
\centering
\fboxrule=2pt{
	\fbox
	{
		\begin{minipage}{\linewidth}
			\centering
			#2
		\end{minipage}
}}

\newpage
%---------- End page ----------

} 

% ---------- Starting Document ----------
\begin{document}
	
	\title{\textbf{\centering{Laboratorio creativo per bambini}\\Trova le opere dentro il museo.}}
	\author{Francesco Rombaldoni}
	\date{}
	
	\maketitle
	\newpage
	
	\tableofcontents
	\newpage
	
	\section{Come giocare}
	\begin{center}
		\textbf{Le regole sono rivolte agli operatori.}
	\end{center}
	
	\subsection{Variante 1}
	Dopo aver ritagliato le immagini, consegnarle a rotazione ai bambini fino ad esaurire il mazzo, aggiungendo qualora ne fossero sprovvisti, delle penne con le quali poter scrivere sulla carta.\\
	A partire dal suggerimento fornito dalla porzione d'immagine che compone la carta, i bambini dovranno girare liberamente nel museo per ritrovare l'opera, e ricopiare la descrizione della suddetta nel riquadro posto sotto la porzione d'immagine.\\
	Quando tutti i bambini hanno finito di completare il compito, procedere con la correzione spiegata, concentrandosi in caso di errore sulle differenze tra l'opera da individuare e quella riportata dal bambino.
	
	\subsection{Variante 2}
	Dopo aver ritagliato le immagini, consegnarle a rotazione ai bambini fino ad esaurire il mazzo, aggiungendo qualora ne fossero sprovvisti, delle penne con le quali poter scrivere sulla carta.\\
	A partire dal suggerimento fornito dalla porzione d'immagine che compone la carta, i bambini dovranno esercitare la propria memoria cercando d'individuare l'opera a partire dal suggerimento e scrivendo nel riquadro sottostante quante più informazioni possibili sull'opera.\\
	Far scrivere a tutti i bambini i loro nomi su una estremità della carte possedute e successivamente raccogliere tutte le carte per procedere con la correzione spiegata.\\
	 Muovendosi nel museo, raggiungere assieme i bambini la posizione delle varie opere, e per ogni opera, raggruppare i bambini che hanno dovuto descriverla (i loro nomi sono scritti sulle carte), leggere quello che hanno scritto, per poi fare il confronto con la suddetta.
	
	
	\vspace*{\fill}
	\centering
	\fboxrule=2pt
	\fbox
	{
		\begin{minipage}{\linewidth}
			In caso di dubbi per la correzione, tenere una copia digitale di questo documento consultabile dallo "smartphone". Nella sezione "Immagini e didascalie" ogni immagine frammentata è presentata con la relativa didascalia posta inferiormente.
		\end{minipage}
	}

	\newpage
	\section{Gestione immagini nel documento}
	
	\begin{tikzpicture}
		\node[anchor=south west,inner sep=0] (image) at (0,0) {\includegraphics[scale=0.85]{example-image}};
		\begin{scope}[x={(image.south east)},y={(image.north west)}]
			\draw[help lines,xstep=.1,ystep=.1] (0,0) grid (1,1);
			\foreach \x in {0,1,...,9} { \node [anchor=north] at (\x/10,0) {0.\x}; }
			\foreach \y in {0,1,...,9} { \node [anchor=east] at (0,\y/10) {0.\y}; }
		\end{scope}
	\end{tikzpicture}
	\newline
	
	\subsection{Proprietà d'immagine}
	\begin{itemize}
		\item L'immagine è divisa in: 100 parti.
		\item Numerazione parti: da 0 a 99
		\item Percentuale di visione d'immagine scelta: 30\% (mostrate 30 parti su 100).
		\item Sequenza di parti visibili prima immagine:  6 8 9 12 13 22 25 28 29 31 32 34 36 38 40 41 42 45 48 50 53 57 69 70 75 76 77 79 81 85.
		\item Sequenza di parti visibili seconda immagine:  1 4 6 8 9 18 22 24 25 27 28 29 30 37 39 51 52 54 57 59 63 65 70 74 77 79 81 82 85 94.
		\item Sequenza di parti visibili terza immagine:  5 7 18 20 21 24 26 27 28 29 32 35 40 43 46 48 49 54 60 61 62 68 71 79 81 84 90 94 97 99.
		\item Sequenza di parti visibili quarta immagine:  1 4 7 12 15 16 17 21 24 25 26 28 34 37 40 43 49 51 54 56 61 67 69 76 82 83 86 91 95 99.
	\end{itemize}
	\textit{Ogni sequenza numerica è stata generata con l'ausilio di un generatore di sequenze numeriche casuali.}

	\newpage
	\section{Immagini e didascalie}
	
	\cards{\includegraphics[]{Mengaroni_Ferruccio-Medusa.jpg}} {Mengaroni Ferruccio - Medusa.}
	
	\cards{\includegraphics[scale=0.047]{Bellini_Giovanni-Incoronazione_della_Vergine.jpg}}{Bellini Giovanni - Incoronazione della Vergine.}
	
	\cards{\includegraphics[scale=0.034]{Vitale_da_Bologna-Santo_Ambrogio_in_trono.jpg}}{Vitale da Bologna - Sant'Ambrogio in trono.}
	
	\cards{\includegraphics[scale=0.04]{Desani_Pietro-Rebecca_ed_Eleazar.jpg}}{Desani Pietro - Rebecca ed Eleazar.}
	
	\cards{\includegraphics[]{Giovanni_Antonio_Garella-Leda_e_il_cigno.jpg}}{Leda e il Cigno (Giove) di fattura ad opera di Giovanni Antonio Garella.}
	
	\cards{\includegraphics[scale=0.18]{Specchio_di_Murano.jpg}}{Specchio in vetro di Murano.}
	
	\cards{\includegraphics[scale=0.4]{Vedute_di_Roma_1.jpg}}{Stipo con vedute di Roma.}
		
	\cards{\includegraphics[scale=0.15]{Orologio_notturno.jpg}}{Orologio notturno.}
	
	\cards{\includegraphics[scale=0.8]{Scacciani_Antonio-Vassoio-Rosa.jpg}}{Scacciani Antonio - Vassoio - Rosa.}
	
	\cards{\includegraphics[scale=0.034]{Milani_Aureliano-Mercato.jpg}}{Milani Aureliano - Mercato.}
	
	\cards{\includegraphics[scale=0.049]{Berentz_Christian-Fiori_e_frutta_con_bicchieri_di_cristallo.jpg}}{Berentz Christian - Fiori e frutta con bicchieri di cristallo.}
	
	\cards{\includegraphics[scale=0.051]{Gianlisi_Antonio_Junior-Trompe_l_oeil_con_sonetto_in_onore_di_Eugenio_di_Savoia_e_mensola_con_oggetti.jpg}}{Gianlisi Antonio Junior - Trompe l'oeil con sonetto in onore di Eugenio di Savoia e mensola con oggetti.}
	
	\cards{\includegraphics[scale=0.050]{Gianlisi_Antonio_Junior-Trompe_l_oeil_con_paesaggio_forbici_e_mensola_con_oggetti.jpg}}{Gianlisi Antonio Junior - Trompe l'oeil con paesaggio forbici e mensola con oggetti.}
	
	\cards{\includegraphics[scale=0.7]{Realfonzo_Tommaso-Natura_morta_con_dolci_frutta_uova_e_formaggi.jpg}}{Realfonzo Tommaso - Natura morta con dolci frutta uova e formaggi.}
	
	\cards{\includegraphics[scale=0.05]{Gessi_Giovan_Francesco-Morte_di_Adone.jpg}}{Gessi Giovan Francesco - Morte di Adone.}
	
	\vspace*{\fill}
	% Print license shield
	\doclicenseThis
	
\end{document}	
